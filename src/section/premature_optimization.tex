\begin{frame}
	\frametitle{Avoid premature optimization}
	\begin{itemize}[<+-| highlight@+>]
		\item Writing optimized code takes longer.
		\item Optimized code is more complex, thus less readable and harder to maintain.
		\item Optimized code often introduces dependencies.
		\item Adds no value without an actual performance need.
		\item Most operations aren't CPU-bound. Optimizing a particular piece of code may have a neglectable impact on overall performance.
		\item Developers are notoriously bad at estimating what to optimize.
		\item Compilers and interpreters are far better at optimizing code. Some seeming optimizations may even block the compiler/interpreter from applying it's own optimizations.
		\item Beware: Don't pessimize prematurely! If it doesn't take longer to implement and doesn't reduce code clarity, please do it.
	\end{itemize}
\end{frame}


\begin{frame}
	\frametitle{How to optimize code, when the need is proven}
	\begin{itemize}[<+-| highlight@+>]
		\item Measure first.
		\item Define optimization goals.
		\item Look for algorithmic optimization first.
		\item Try to encapsulate and modularize the optimization.
		\item Write a comment explaining the reason of the optimization.
		\item Keep the unoptimized code, both as a reference and to write correctness tests against.
		\item Run the optimized code in different scenarios, not only in the scenario you're trying to optimize.
	\end{itemize}
\end{frame}

